
% Default to the notebook output style

    


% Inherit from the specified cell style.




    
\documentclass[11pt]{article}

    
    
    \usepackage[T1]{fontenc}
    % Nicer default font (+ math font) than Computer Modern for most use cases
    \usepackage{mathpazo}

    % Basic figure setup, for now with no caption control since it's done
    % automatically by Pandoc (which extracts ![](path) syntax from Markdown).
    \usepackage{graphicx}
    % We will generate all images so they have a width \maxwidth. This means
    % that they will get their normal width if they fit onto the page, but
    % are scaled down if they would overflow the margins.
    \makeatletter
    \def\maxwidth{\ifdim\Gin@nat@width>\linewidth\linewidth
    \else\Gin@nat@width\fi}
    \makeatother
    \let\Oldincludegraphics\includegraphics
    % Set max figure width to be 80% of text width, for now hardcoded.
    \renewcommand{\includegraphics}[1]{\Oldincludegraphics[width=.8\maxwidth]{#1}}
    % Ensure that by default, figures have no caption (until we provide a
    % proper Figure object with a Caption API and a way to capture that
    % in the conversion process - todo).
    \usepackage{caption}
    \DeclareCaptionLabelFormat{nolabel}{}
    \captionsetup{labelformat=nolabel}

    \usepackage{adjustbox} % Used to constrain images to a maximum size 
    \usepackage{xcolor} % Allow colors to be defined
    \usepackage{enumerate} % Needed for markdown enumerations to work
    \usepackage{geometry} % Used to adjust the document margins
    \usepackage{amsmath} % Equations
    \usepackage{amssymb} % Equations
    \usepackage{textcomp} % defines textquotesingle
    % Hack from http://tex.stackexchange.com/a/47451/13684:
    \AtBeginDocument{%
        \def\PYZsq{\textquotesingle}% Upright quotes in Pygmentized code
    }
    \usepackage{upquote} % Upright quotes for verbatim code
    \usepackage{eurosym} % defines \euro
    \usepackage[mathletters]{ucs} % Extended unicode (utf-8) support
    \usepackage[utf8x]{inputenc} % Allow utf-8 characters in the tex document
    \usepackage{fancyvrb} % verbatim replacement that allows latex
    \usepackage{grffile} % extends the file name processing of package graphics 
                         % to support a larger range 
    % The hyperref package gives us a pdf with properly built
    % internal navigation ('pdf bookmarks' for the table of contents,
    % internal cross-reference links, web links for URLs, etc.)
    \usepackage{hyperref}
    \usepackage{longtable} % longtable support required by pandoc >1.10
    \usepackage{booktabs}  % table support for pandoc > 1.12.2
    \usepackage[inline]{enumitem} % IRkernel/repr support (it uses the enumerate* environment)
    \usepackage[normalem]{ulem} % ulem is needed to support strikethroughs (\sout)
                                % normalem makes italics be italics, not underlines
    

    
    
    % Colors for the hyperref package
    \definecolor{urlcolor}{rgb}{0,.145,.698}
    \definecolor{linkcolor}{rgb}{.71,0.21,0.01}
    \definecolor{citecolor}{rgb}{.12,.54,.11}

    % ANSI colors
    \definecolor{ansi-black}{HTML}{3E424D}
    \definecolor{ansi-black-intense}{HTML}{282C36}
    \definecolor{ansi-red}{HTML}{E75C58}
    \definecolor{ansi-red-intense}{HTML}{B22B31}
    \definecolor{ansi-green}{HTML}{00A250}
    \definecolor{ansi-green-intense}{HTML}{007427}
    \definecolor{ansi-yellow}{HTML}{DDB62B}
    \definecolor{ansi-yellow-intense}{HTML}{B27D12}
    \definecolor{ansi-blue}{HTML}{208FFB}
    \definecolor{ansi-blue-intense}{HTML}{0065CA}
    \definecolor{ansi-magenta}{HTML}{D160C4}
    \definecolor{ansi-magenta-intense}{HTML}{A03196}
    \definecolor{ansi-cyan}{HTML}{60C6C8}
    \definecolor{ansi-cyan-intense}{HTML}{258F8F}
    \definecolor{ansi-white}{HTML}{C5C1B4}
    \definecolor{ansi-white-intense}{HTML}{A1A6B2}

    % commands and environments needed by pandoc snippets
    % extracted from the output of `pandoc -s`
    \providecommand{\tightlist}{%
      \setlength{\itemsep}{0pt}\setlength{\parskip}{0pt}}
    \DefineVerbatimEnvironment{Highlighting}{Verbatim}{commandchars=\\\{\}}
    % Add ',fontsize=\small' for more characters per line
    \newenvironment{Shaded}{}{}
    \newcommand{\KeywordTok}[1]{\textcolor[rgb]{0.00,0.44,0.13}{\textbf{{#1}}}}
    \newcommand{\DataTypeTok}[1]{\textcolor[rgb]{0.56,0.13,0.00}{{#1}}}
    \newcommand{\DecValTok}[1]{\textcolor[rgb]{0.25,0.63,0.44}{{#1}}}
    \newcommand{\BaseNTok}[1]{\textcolor[rgb]{0.25,0.63,0.44}{{#1}}}
    \newcommand{\FloatTok}[1]{\textcolor[rgb]{0.25,0.63,0.44}{{#1}}}
    \newcommand{\CharTok}[1]{\textcolor[rgb]{0.25,0.44,0.63}{{#1}}}
    \newcommand{\StringTok}[1]{\textcolor[rgb]{0.25,0.44,0.63}{{#1}}}
    \newcommand{\CommentTok}[1]{\textcolor[rgb]{0.38,0.63,0.69}{\textit{{#1}}}}
    \newcommand{\OtherTok}[1]{\textcolor[rgb]{0.00,0.44,0.13}{{#1}}}
    \newcommand{\AlertTok}[1]{\textcolor[rgb]{1.00,0.00,0.00}{\textbf{{#1}}}}
    \newcommand{\FunctionTok}[1]{\textcolor[rgb]{0.02,0.16,0.49}{{#1}}}
    \newcommand{\RegionMarkerTok}[1]{{#1}}
    \newcommand{\ErrorTok}[1]{\textcolor[rgb]{1.00,0.00,0.00}{\textbf{{#1}}}}
    \newcommand{\NormalTok}[1]{{#1}}
    
    % Additional commands for more recent versions of Pandoc
    \newcommand{\ConstantTok}[1]{\textcolor[rgb]{0.53,0.00,0.00}{{#1}}}
    \newcommand{\SpecialCharTok}[1]{\textcolor[rgb]{0.25,0.44,0.63}{{#1}}}
    \newcommand{\VerbatimStringTok}[1]{\textcolor[rgb]{0.25,0.44,0.63}{{#1}}}
    \newcommand{\SpecialStringTok}[1]{\textcolor[rgb]{0.73,0.40,0.53}{{#1}}}
    \newcommand{\ImportTok}[1]{{#1}}
    \newcommand{\DocumentationTok}[1]{\textcolor[rgb]{0.73,0.13,0.13}{\textit{{#1}}}}
    \newcommand{\AnnotationTok}[1]{\textcolor[rgb]{0.38,0.63,0.69}{\textbf{\textit{{#1}}}}}
    \newcommand{\CommentVarTok}[1]{\textcolor[rgb]{0.38,0.63,0.69}{\textbf{\textit{{#1}}}}}
    \newcommand{\VariableTok}[1]{\textcolor[rgb]{0.10,0.09,0.49}{{#1}}}
    \newcommand{\ControlFlowTok}[1]{\textcolor[rgb]{0.00,0.44,0.13}{\textbf{{#1}}}}
    \newcommand{\OperatorTok}[1]{\textcolor[rgb]{0.40,0.40,0.40}{{#1}}}
    \newcommand{\BuiltInTok}[1]{{#1}}
    \newcommand{\ExtensionTok}[1]{{#1}}
    \newcommand{\PreprocessorTok}[1]{\textcolor[rgb]{0.74,0.48,0.00}{{#1}}}
    \newcommand{\AttributeTok}[1]{\textcolor[rgb]{0.49,0.56,0.16}{{#1}}}
    \newcommand{\InformationTok}[1]{\textcolor[rgb]{0.38,0.63,0.69}{\textbf{\textit{{#1}}}}}
    \newcommand{\WarningTok}[1]{\textcolor[rgb]{0.38,0.63,0.69}{\textbf{\textit{{#1}}}}}
    
    
    % Define a nice break command that doesn't care if a line doesn't already
    % exist.
    \def\br{\hspace*{\fill} \\* }
    % Math Jax compatability definitions
    \def\gt{>}
    \def\lt{<}
    % Document parameters
    \title{DMO Assignment 2 - Programming}
    
    
    

    % Pygments definitions
    
\makeatletter
\def\PY@reset{\let\PY@it=\relax \let\PY@bf=\relax%
    \let\PY@ul=\relax \let\PY@tc=\relax%
    \let\PY@bc=\relax \let\PY@ff=\relax}
\def\PY@tok#1{\csname PY@tok@#1\endcsname}
\def\PY@toks#1+{\ifx\relax#1\empty\else%
    \PY@tok{#1}\expandafter\PY@toks\fi}
\def\PY@do#1{\PY@bc{\PY@tc{\PY@ul{%
    \PY@it{\PY@bf{\PY@ff{#1}}}}}}}
\def\PY#1#2{\PY@reset\PY@toks#1+\relax+\PY@do{#2}}

\expandafter\def\csname PY@tok@w\endcsname{\def\PY@tc##1{\textcolor[rgb]{0.73,0.73,0.73}{##1}}}
\expandafter\def\csname PY@tok@c\endcsname{\let\PY@it=\textit\def\PY@tc##1{\textcolor[rgb]{0.25,0.50,0.50}{##1}}}
\expandafter\def\csname PY@tok@cp\endcsname{\def\PY@tc##1{\textcolor[rgb]{0.74,0.48,0.00}{##1}}}
\expandafter\def\csname PY@tok@k\endcsname{\let\PY@bf=\textbf\def\PY@tc##1{\textcolor[rgb]{0.00,0.50,0.00}{##1}}}
\expandafter\def\csname PY@tok@kp\endcsname{\def\PY@tc##1{\textcolor[rgb]{0.00,0.50,0.00}{##1}}}
\expandafter\def\csname PY@tok@kt\endcsname{\def\PY@tc##1{\textcolor[rgb]{0.69,0.00,0.25}{##1}}}
\expandafter\def\csname PY@tok@o\endcsname{\def\PY@tc##1{\textcolor[rgb]{0.40,0.40,0.40}{##1}}}
\expandafter\def\csname PY@tok@ow\endcsname{\let\PY@bf=\textbf\def\PY@tc##1{\textcolor[rgb]{0.67,0.13,1.00}{##1}}}
\expandafter\def\csname PY@tok@nb\endcsname{\def\PY@tc##1{\textcolor[rgb]{0.00,0.50,0.00}{##1}}}
\expandafter\def\csname PY@tok@nf\endcsname{\def\PY@tc##1{\textcolor[rgb]{0.00,0.00,1.00}{##1}}}
\expandafter\def\csname PY@tok@nc\endcsname{\let\PY@bf=\textbf\def\PY@tc##1{\textcolor[rgb]{0.00,0.00,1.00}{##1}}}
\expandafter\def\csname PY@tok@nn\endcsname{\let\PY@bf=\textbf\def\PY@tc##1{\textcolor[rgb]{0.00,0.00,1.00}{##1}}}
\expandafter\def\csname PY@tok@ne\endcsname{\let\PY@bf=\textbf\def\PY@tc##1{\textcolor[rgb]{0.82,0.25,0.23}{##1}}}
\expandafter\def\csname PY@tok@nv\endcsname{\def\PY@tc##1{\textcolor[rgb]{0.10,0.09,0.49}{##1}}}
\expandafter\def\csname PY@tok@no\endcsname{\def\PY@tc##1{\textcolor[rgb]{0.53,0.00,0.00}{##1}}}
\expandafter\def\csname PY@tok@nl\endcsname{\def\PY@tc##1{\textcolor[rgb]{0.63,0.63,0.00}{##1}}}
\expandafter\def\csname PY@tok@ni\endcsname{\let\PY@bf=\textbf\def\PY@tc##1{\textcolor[rgb]{0.60,0.60,0.60}{##1}}}
\expandafter\def\csname PY@tok@na\endcsname{\def\PY@tc##1{\textcolor[rgb]{0.49,0.56,0.16}{##1}}}
\expandafter\def\csname PY@tok@nt\endcsname{\let\PY@bf=\textbf\def\PY@tc##1{\textcolor[rgb]{0.00,0.50,0.00}{##1}}}
\expandafter\def\csname PY@tok@nd\endcsname{\def\PY@tc##1{\textcolor[rgb]{0.67,0.13,1.00}{##1}}}
\expandafter\def\csname PY@tok@s\endcsname{\def\PY@tc##1{\textcolor[rgb]{0.73,0.13,0.13}{##1}}}
\expandafter\def\csname PY@tok@sd\endcsname{\let\PY@it=\textit\def\PY@tc##1{\textcolor[rgb]{0.73,0.13,0.13}{##1}}}
\expandafter\def\csname PY@tok@si\endcsname{\let\PY@bf=\textbf\def\PY@tc##1{\textcolor[rgb]{0.73,0.40,0.53}{##1}}}
\expandafter\def\csname PY@tok@se\endcsname{\let\PY@bf=\textbf\def\PY@tc##1{\textcolor[rgb]{0.73,0.40,0.13}{##1}}}
\expandafter\def\csname PY@tok@sr\endcsname{\def\PY@tc##1{\textcolor[rgb]{0.73,0.40,0.53}{##1}}}
\expandafter\def\csname PY@tok@ss\endcsname{\def\PY@tc##1{\textcolor[rgb]{0.10,0.09,0.49}{##1}}}
\expandafter\def\csname PY@tok@sx\endcsname{\def\PY@tc##1{\textcolor[rgb]{0.00,0.50,0.00}{##1}}}
\expandafter\def\csname PY@tok@m\endcsname{\def\PY@tc##1{\textcolor[rgb]{0.40,0.40,0.40}{##1}}}
\expandafter\def\csname PY@tok@gh\endcsname{\let\PY@bf=\textbf\def\PY@tc##1{\textcolor[rgb]{0.00,0.00,0.50}{##1}}}
\expandafter\def\csname PY@tok@gu\endcsname{\let\PY@bf=\textbf\def\PY@tc##1{\textcolor[rgb]{0.50,0.00,0.50}{##1}}}
\expandafter\def\csname PY@tok@gd\endcsname{\def\PY@tc##1{\textcolor[rgb]{0.63,0.00,0.00}{##1}}}
\expandafter\def\csname PY@tok@gi\endcsname{\def\PY@tc##1{\textcolor[rgb]{0.00,0.63,0.00}{##1}}}
\expandafter\def\csname PY@tok@gr\endcsname{\def\PY@tc##1{\textcolor[rgb]{1.00,0.00,0.00}{##1}}}
\expandafter\def\csname PY@tok@ge\endcsname{\let\PY@it=\textit}
\expandafter\def\csname PY@tok@gs\endcsname{\let\PY@bf=\textbf}
\expandafter\def\csname PY@tok@gp\endcsname{\let\PY@bf=\textbf\def\PY@tc##1{\textcolor[rgb]{0.00,0.00,0.50}{##1}}}
\expandafter\def\csname PY@tok@go\endcsname{\def\PY@tc##1{\textcolor[rgb]{0.53,0.53,0.53}{##1}}}
\expandafter\def\csname PY@tok@gt\endcsname{\def\PY@tc##1{\textcolor[rgb]{0.00,0.27,0.87}{##1}}}
\expandafter\def\csname PY@tok@err\endcsname{\def\PY@bc##1{\setlength{\fboxsep}{0pt}\fcolorbox[rgb]{1.00,0.00,0.00}{1,1,1}{\strut ##1}}}
\expandafter\def\csname PY@tok@kc\endcsname{\let\PY@bf=\textbf\def\PY@tc##1{\textcolor[rgb]{0.00,0.50,0.00}{##1}}}
\expandafter\def\csname PY@tok@kd\endcsname{\let\PY@bf=\textbf\def\PY@tc##1{\textcolor[rgb]{0.00,0.50,0.00}{##1}}}
\expandafter\def\csname PY@tok@kn\endcsname{\let\PY@bf=\textbf\def\PY@tc##1{\textcolor[rgb]{0.00,0.50,0.00}{##1}}}
\expandafter\def\csname PY@tok@kr\endcsname{\let\PY@bf=\textbf\def\PY@tc##1{\textcolor[rgb]{0.00,0.50,0.00}{##1}}}
\expandafter\def\csname PY@tok@bp\endcsname{\def\PY@tc##1{\textcolor[rgb]{0.00,0.50,0.00}{##1}}}
\expandafter\def\csname PY@tok@fm\endcsname{\def\PY@tc##1{\textcolor[rgb]{0.00,0.00,1.00}{##1}}}
\expandafter\def\csname PY@tok@vc\endcsname{\def\PY@tc##1{\textcolor[rgb]{0.10,0.09,0.49}{##1}}}
\expandafter\def\csname PY@tok@vg\endcsname{\def\PY@tc##1{\textcolor[rgb]{0.10,0.09,0.49}{##1}}}
\expandafter\def\csname PY@tok@vi\endcsname{\def\PY@tc##1{\textcolor[rgb]{0.10,0.09,0.49}{##1}}}
\expandafter\def\csname PY@tok@vm\endcsname{\def\PY@tc##1{\textcolor[rgb]{0.10,0.09,0.49}{##1}}}
\expandafter\def\csname PY@tok@sa\endcsname{\def\PY@tc##1{\textcolor[rgb]{0.73,0.13,0.13}{##1}}}
\expandafter\def\csname PY@tok@sb\endcsname{\def\PY@tc##1{\textcolor[rgb]{0.73,0.13,0.13}{##1}}}
\expandafter\def\csname PY@tok@sc\endcsname{\def\PY@tc##1{\textcolor[rgb]{0.73,0.13,0.13}{##1}}}
\expandafter\def\csname PY@tok@dl\endcsname{\def\PY@tc##1{\textcolor[rgb]{0.73,0.13,0.13}{##1}}}
\expandafter\def\csname PY@tok@s2\endcsname{\def\PY@tc##1{\textcolor[rgb]{0.73,0.13,0.13}{##1}}}
\expandafter\def\csname PY@tok@sh\endcsname{\def\PY@tc##1{\textcolor[rgb]{0.73,0.13,0.13}{##1}}}
\expandafter\def\csname PY@tok@s1\endcsname{\def\PY@tc##1{\textcolor[rgb]{0.73,0.13,0.13}{##1}}}
\expandafter\def\csname PY@tok@mb\endcsname{\def\PY@tc##1{\textcolor[rgb]{0.40,0.40,0.40}{##1}}}
\expandafter\def\csname PY@tok@mf\endcsname{\def\PY@tc##1{\textcolor[rgb]{0.40,0.40,0.40}{##1}}}
\expandafter\def\csname PY@tok@mh\endcsname{\def\PY@tc##1{\textcolor[rgb]{0.40,0.40,0.40}{##1}}}
\expandafter\def\csname PY@tok@mi\endcsname{\def\PY@tc##1{\textcolor[rgb]{0.40,0.40,0.40}{##1}}}
\expandafter\def\csname PY@tok@il\endcsname{\def\PY@tc##1{\textcolor[rgb]{0.40,0.40,0.40}{##1}}}
\expandafter\def\csname PY@tok@mo\endcsname{\def\PY@tc##1{\textcolor[rgb]{0.40,0.40,0.40}{##1}}}
\expandafter\def\csname PY@tok@ch\endcsname{\let\PY@it=\textit\def\PY@tc##1{\textcolor[rgb]{0.25,0.50,0.50}{##1}}}
\expandafter\def\csname PY@tok@cm\endcsname{\let\PY@it=\textit\def\PY@tc##1{\textcolor[rgb]{0.25,0.50,0.50}{##1}}}
\expandafter\def\csname PY@tok@cpf\endcsname{\let\PY@it=\textit\def\PY@tc##1{\textcolor[rgb]{0.25,0.50,0.50}{##1}}}
\expandafter\def\csname PY@tok@c1\endcsname{\let\PY@it=\textit\def\PY@tc##1{\textcolor[rgb]{0.25,0.50,0.50}{##1}}}
\expandafter\def\csname PY@tok@cs\endcsname{\let\PY@it=\textit\def\PY@tc##1{\textcolor[rgb]{0.25,0.50,0.50}{##1}}}

\def\PYZbs{\char`\\}
\def\PYZus{\char`\_}
\def\PYZob{\char`\{}
\def\PYZcb{\char`\}}
\def\PYZca{\char`\^}
\def\PYZam{\char`\&}
\def\PYZlt{\char`\<}
\def\PYZgt{\char`\>}
\def\PYZsh{\char`\#}
\def\PYZpc{\char`\%}
\def\PYZdl{\char`\$}
\def\PYZhy{\char`\-}
\def\PYZsq{\char`\'}
\def\PYZdq{\char`\"}
\def\PYZti{\char`\~}
% for compatibility with earlier versions
\def\PYZat{@}
\def\PYZlb{[}
\def\PYZrb{]}
\makeatother


    % Exact colors from NB
    \definecolor{incolor}{rgb}{0.0, 0.0, 0.5}
    \definecolor{outcolor}{rgb}{0.545, 0.0, 0.0}



    
    % Prevent overflowing lines due to hard-to-break entities
    \sloppy 
    % Setup hyperref package
    \hypersetup{
      breaklinks=true,  % so long urls are correctly broken across lines
      colorlinks=true,
      urlcolor=urlcolor,
      linkcolor=linkcolor,
      citecolor=citecolor,
      }
    % Slightly bigger margins than the latex defaults
    
    \geometry{verbose,tmargin=1in,bmargin=1in,lmargin=1in,rmargin=1in}
    
    

    \begin{document}
    
    
    \maketitle
    
    

    
    Hannah Busshoff, Snorri Petersen and Sebastian Wolf

\section*{Table of Content}\label{table-of-content}
\addcontentsline{toc}{section}{Table of Content}

\begin{enumerate}
\def\labelenumi{\arabic{enumi}.}
\tightlist
\item
  Edit Distance \newline
  1.1 Discussion of the problem \newline
  1.2 Proposed algorithm \newline
  1.3 Proof of Correctness
\item
  Huffman Codes \newline
  2.1 Discussion of the problem \newline
  2.2 Proposed algorithm \newline
  2.3 Proof of Correctness
\item
  References
\end{enumerate}

    \section{Edit Distance}\label{edit-distance}

\subsection{Discussion of the problem}\label{discussion-of-the-problem}

The edit distance is a measure for the similarity of two strings. It
counts the minimum number of operations required to transform one string
into another. The operations considered for the edit distance are:

\begin{enumerate}
\def\labelenumi{\arabic{enumi}.}
\tightlist
\item
  Insert a character
\item
  Delete an existing character
\item
  Substitute a character by another
\end{enumerate}

When the edit distance is measured using only these three operations it
is also called a 'Levensthein distance'. The edit distance has many
applications, including automatic spelling correction in natural
language processing or the comparison of DNA in bioinformatics.

In this assignment we are presented with a problem of assessing the
minimum operations between two sets of two text strings:

\begin{enumerate}
\def\labelenumi{\arabic{enumi}.}
\tightlist
\item
  DNA where

  \begin{enumerate}
  \def\labelenumii{\alph{enumii})}
  \tightlist
  \item
    X = ACTACTAGATTACTTACGGATCAGGTACTTTAGAGGCTTGCAACCAY
  \item
    Y = TACTAGCTTACTTACCCATCAGGTTTTAGAGATGGCAACCA
  \end{enumerate}
\item
  Proteins where

  \begin{enumerate}
  \def\labelenumii{\alph{enumii})}
  \tightlist
  \item
    X = AASRPRSGVPAQSDSDPCQNLAATPIPSRPPSSQSCQKCRADARQGRWGPY
  \item
    Y = SGAPGQRGEPGPQGHAGAPGPPGPPGSDG
  \end{enumerate}
\end{enumerate}

There are potentially many ways to transform one string into another,
and trying out all possible combinations to find the lowest cost option
would be prohibitively expensive. To solve the problem more efficiently,
we therefore propose an algorithm based on the solution for the longest
common subsequence that was covered in class. This algortihm falls under
the dynamic programming paradigm.

The idea of Dynamic Programming is to solve a large problem that has too
many potential combinations to solve greedily, by dividing it into
subproblems. In contrast to divide and conquer algorithms, the
subproblems in dynamic programming overlap; the solution of one
subproblem depends on the solution of others.

The subproblems in this case involves comparing substrings, rather than
the whole string at once. The program creates a table where the two
strings are compared prefix by prefix where all three operations
(insert, delete and substitude) are available. Here below each operation
comes with the cost of 1.

    \subsection{Proposed algorithm}\label{proposed-algorithm}

Define a function \textbf{\emph{'compare'}}, which gives for a given
cell D{[}m,n{]} of the matrix D, where m denotes the row index and n the
column index, the optimal prior cell that can lead the algorithm to
D{[}m,n{]} by either deletion, insertion, substitution or no
manipulations. Optimality is defined in terms of minimal value of the
available prior cell. If there is a tie between prior cells, the
algorithm always takes the diagonal.

    \begin{Verbatim}[commandchars=\\\{\}]
{\color{incolor}In [{\color{incolor}1}]:} \PY{c+c1}{\PYZsh{} Import numpy package to create an array.}
        \PY{k+kn}{import} \PY{n+nn}{numpy} \PY{k}{as} \PY{n+nn}{np}
        
        \PY{k}{def} \PY{n+nf}{compare}\PY{p}{(}\PY{n}{D}\PY{p}{,} \PY{n}{m}\PY{p}{,} \PY{n}{n}\PY{p}{,} \PY{n}{cost1}\PY{p}{,} \PY{n}{cost2}\PY{p}{)}\PY{p}{:}
            \PY{k}{if} \PY{p}{(}\PY{n}{m} \PY{o}{==} \PY{l+m+mi}{0}\PY{p}{)} \PY{o+ow}{and} \PY{p}{(}\PY{n}{n} \PY{o}{\PYZgt{}} \PY{l+m+mi}{0}\PY{p}{)}\PY{p}{:}
                \PY{k}{return} \PY{l+s+s2}{\PYZdq{}}\PY{l+s+s2}{I}\PY{l+s+s2}{\PYZdq{}}\PY{p}{,} \PY{n+nb}{max}\PY{p}{(}\PY{n}{m}\PY{p}{,}\PY{l+m+mi}{0}\PY{p}{)}\PY{p}{,} \PY{n+nb}{max}\PY{p}{(}\PY{n}{n}\PY{o}{\PYZhy{}}\PY{l+m+mi}{1}\PY{p}{,}\PY{l+m+mi}{0}\PY{p}{)}
            \PY{k}{elif} \PY{p}{(}\PY{n}{n} \PY{o}{==} \PY{l+m+mi}{0}\PY{p}{)} \PY{o+ow}{and} \PY{p}{(}\PY{n}{m} \PY{o}{\PYZgt{}} \PY{l+m+mi}{0}\PY{p}{)}\PY{p}{:}
                \PY{k}{return} \PY{l+s+s2}{\PYZdq{}}\PY{l+s+s2}{D}\PY{l+s+s2}{\PYZdq{}}\PY{p}{,} \PY{n+nb}{max}\PY{p}{(}\PY{n}{m}\PY{o}{\PYZhy{}}\PY{l+m+mi}{1}\PY{p}{,}\PY{l+m+mi}{0}\PY{p}{)}\PY{p}{,} \PY{n+nb}{max}\PY{p}{(}\PY{n}{n}\PY{p}{,}\PY{l+m+mi}{0}\PY{p}{)}
            \PY{k}{elif} \PY{p}{(}\PY{n}{D}\PY{p}{[}\PY{n}{m} \PY{o}{\PYZhy{}} \PY{l+m+mi}{1}\PY{p}{,} \PY{n}{n} \PY{o}{\PYZhy{}} \PY{l+m+mi}{1}\PY{p}{]} \PY{o}{==} \PY{n}{D}\PY{p}{[}\PY{n}{m}\PY{p}{,}\PY{n}{n}\PY{p}{]}\PY{p}{)} \PY{o+ow}{and} \PY{p}{(}\PY{n}{D}\PY{p}{[}\PY{n}{m} \PY{o}{\PYZhy{}} \PY{l+m+mi}{1}\PY{p}{,} \PY{n}{n}\PY{o}{\PYZhy{}}\PY{l+m+mi}{1}\PY{p}{]} \PY{o}{\PYZlt{}}\PY{o}{=} \PY{n+nb}{min}\PY{p}{(}\PY{n}{D}\PY{p}{[}\PY{n}{m}\PY{p}{,} \PY{n}{n}\PY{o}{\PYZhy{}}\PY{l+m+mi}{1}
                \PY{p}{]} \PY{o}{+} \PY{n}{cost1}\PY{p}{,} \PY{n}{D}\PY{p}{[}\PY{n}{m}\PY{o}{\PYZhy{}}\PY{l+m+mi}{1}\PY{p}{,} \PY{n}{n}\PY{p}{]} \PY{o}{+} \PY{n}{cost1}\PY{p}{)}\PY{p}{)}\PY{p}{:}
                \PY{k}{return} \PY{l+s+s2}{\PYZdq{}}\PY{l+s+s2}{\PYZhy{}}\PY{l+s+s2}{\PYZdq{}}\PY{p}{,} \PY{n+nb}{max}\PY{p}{(}\PY{n}{m}\PY{o}{\PYZhy{}}\PY{l+m+mi}{1}\PY{p}{,}\PY{l+m+mi}{0}\PY{p}{)}\PY{p}{,} \PY{n+nb}{max}\PY{p}{(}\PY{n}{n}\PY{o}{\PYZhy{}}\PY{l+m+mi}{1}\PY{p}{,}\PY{l+m+mi}{0}\PY{p}{)} \PY{c+c1}{\PYZsh{} nothing}
            \PY{k}{elif} \PY{n}{D}\PY{p}{[}\PY{n}{m} \PY{o}{\PYZhy{}} \PY{l+m+mi}{1}\PY{p}{,} \PY{n}{n} \PY{o}{\PYZhy{}} \PY{l+m+mi}{1}\PY{p}{]} \PY{o}{+} \PY{n}{cost2} \PY{o}{\PYZlt{}}\PY{o}{=} \PY{n+nb}{min}\PY{p}{(}\PY{n}{D}\PY{p}{[}\PY{n}{m}\PY{p}{,} \PY{n}{n}\PY{o}{\PYZhy{}}\PY{l+m+mi}{1}\PY{p}{]}\PY{p}{,} \PY{n}{D}\PY{p}{[}\PY{n}{m}\PY{o}{\PYZhy{}}\PY{l+m+mi}{1}\PY{p}{,} \PY{n}{n}\PY{p}{]}\PY{p}{)} \PY{o}{+} \PY{n}{cost1} \PY{p}{:}
                \PY{k}{return} \PY{l+s+s2}{\PYZdq{}}\PY{l+s+s2}{S}\PY{l+s+s2}{\PYZdq{}}\PY{p}{,} \PY{n+nb}{max}\PY{p}{(}\PY{n}{m}\PY{o}{\PYZhy{}}\PY{l+m+mi}{1}\PY{p}{,}\PY{l+m+mi}{0}\PY{p}{)}\PY{p}{,} \PY{n+nb}{max}\PY{p}{(}\PY{n}{n}\PY{o}{\PYZhy{}}\PY{l+m+mi}{1}\PY{p}{,}\PY{l+m+mi}{0}\PY{p}{)} \PY{c+c1}{\PYZsh{}substitution}
            \PY{k}{elif} \PY{n}{D}\PY{p}{[}\PY{n}{m}\PY{o}{\PYZhy{}}\PY{l+m+mi}{1}\PY{p}{,}\PY{n}{n}\PY{p}{]}  \PY{o}{\PYZlt{}} \PY{n}{D}\PY{p}{[}\PY{n}{m}\PY{p}{,}\PY{n}{n}\PY{o}{\PYZhy{}}\PY{l+m+mi}{1}\PY{p}{]}\PY{p}{:}
                \PY{k}{return} \PY{l+s+s2}{\PYZdq{}}\PY{l+s+s2}{D}\PY{l+s+s2}{\PYZdq{}}\PY{p}{,} \PY{n+nb}{max}\PY{p}{(}\PY{n}{m}\PY{o}{\PYZhy{}}\PY{l+m+mi}{1}\PY{p}{,}\PY{l+m+mi}{0}\PY{p}{)}\PY{p}{,} \PY{n+nb}{max}\PY{p}{(}\PY{n}{n}\PY{p}{,}\PY{l+m+mi}{0}\PY{p}{)} \PY{c+c1}{\PYZsh{} deletion}
            \PY{k}{elif} \PY{n}{D}\PY{p}{[}\PY{n}{m}\PY{p}{,} \PY{n}{n}\PY{o}{\PYZhy{}}\PY{l+m+mi}{1}\PY{p}{]} \PY{o}{\PYZlt{}} \PY{n}{D}\PY{p}{[}\PY{n}{m}\PY{p}{,} \PY{n}{n}\PY{o}{\PYZhy{}}\PY{l+m+mi}{1}\PY{p}{]}\PY{p}{:}
                \PY{k}{return} \PY{l+s+s2}{\PYZdq{}}\PY{l+s+s2}{I}\PY{l+s+s2}{\PYZdq{}}\PY{p}{,} \PY{n+nb}{max}\PY{p}{(}\PY{n}{m}\PY{p}{,}\PY{l+m+mi}{0}\PY{p}{)}\PY{p}{,} \PY{n+nb}{max}\PY{p}{(}\PY{n}{n}\PY{o}{\PYZhy{}}\PY{l+m+mi}{1}\PY{p}{,}\PY{l+m+mi}{0}\PY{p}{)} \PY{c+c1}{\PYZsh{} insertion}
\end{Verbatim}


    Define a function \textbf{\emph{'backtrace'}}. The function recovers the
optimal solution by iteratively applying the compare function. The input
is D{[}m,n{]}, where m denotes the highest row index and n the highest
column index. The backtrace function stops when it reaches D{[}0,0{]}
the starting point of the algorithm.

    \begin{Verbatim}[commandchars=\\\{\}]
{\color{incolor}In [{\color{incolor}2}]:} \PY{k}{def} \PY{n+nf}{backtrace}\PY{p}{(}\PY{n}{D}\PY{p}{,} \PY{n}{m}\PY{p}{,} \PY{n}{n}\PY{p}{,} \PY{n}{cost1}\PY{p}{,} \PY{n}{cost2}\PY{p}{)}\PY{p}{:}
            \PY{n}{changes} \PY{o}{=} \PY{p}{[}\PY{p}{]} \PY{c+c1}{\PYZsh{} Initialize an empty vector to store the backtrace.}
            \PY{c+c1}{\PYZsh{} Iteratively apply the compare function to reach D[0,0], }
            \PY{c+c1}{\PYZsh{} the starting point of the algorithm.}
            \PY{k}{while} \PY{n}{m} \PY{o}{\PYZgt{}} \PY{l+m+mi}{0} \PY{o+ow}{or} \PY{n}{n} \PY{o}{\PYZgt{}} \PY{l+m+mi}{0}\PY{p}{:}
                \PY{c+c1}{\PYZsh{} Compute the optimal prior cell.}
                \PY{n}{result} \PY{o}{=} \PY{n}{compare}\PY{p}{(}\PY{n}{D}\PY{p}{,} \PY{n}{m}\PY{p}{,} \PY{n}{n}\PY{p}{,} \PY{n}{cost1}\PY{p}{,} \PY{n}{cost2}\PY{p}{)} 
                \PY{c+c1}{\PYZsh{} Store the optimal move in the list changes and append it.}
                \PY{n}{changes}\PY{o}{.}\PY{n}{append}\PY{p}{(}\PY{n}{result}\PY{p}{[}\PY{l+m+mi}{0}\PY{p}{]}\PY{p}{)} 
                \PY{n}{m} \PY{o}{=} \PY{n}{result}\PY{p}{[}\PY{l+m+mi}{1}\PY{p}{]} \PY{c+c1}{\PYZsh{} Update row location.}
                \PY{n}{n} \PY{o}{=} \PY{n}{result}\PY{p}{[}\PY{l+m+mi}{2}\PY{p}{]} \PY{c+c1}{\PYZsh{} Update column location.}
            \PY{k}{return} \PY{n}{changes}
\end{Verbatim}


    Define a function \textbf{\emph{'edit\_simple'}}, which computes the
edit distance, with costs of cost1 for insertion and deletion, and cost2
for substitution.

    \begin{Verbatim}[commandchars=\\\{\}]
{\color{incolor}In [{\color{incolor}3}]:} \PY{k}{def} \PY{n+nf}{edit\PYZus{}distance}\PY{p}{(}\PY{n}{str1}\PY{p}{,} \PY{n}{str2}\PY{p}{,} \PY{n}{cost1}\PY{p}{,} \PY{n}{cost2}\PY{p}{)}\PY{p}{:}
            \PY{c+c1}{\PYZsh{}Computing the length of the input strings.}
            \PY{n}{m} \PY{o}{=} \PY{n+nb}{len}\PY{p}{(}\PY{n}{str1}\PY{p}{)}
            \PY{n}{n} \PY{o}{=} \PY{n+nb}{len}\PY{p}{(}\PY{n}{str2}\PY{p}{)}
            \PY{c+c1}{\PYZsh{}Initiliazing the matrix.}
            \PY{n}{D} \PY{o}{=} \PY{n}{np}\PY{o}{.}\PY{n}{zeros}\PY{p}{(}\PY{p}{(}\PY{p}{(}\PY{n}{m}\PY{o}{+}\PY{l+m+mi}{1}\PY{p}{)}\PY{p}{,} \PY{p}{(}\PY{n}{n}\PY{o}{+}\PY{l+m+mi}{1}\PY{p}{)}\PY{p}{)}\PY{p}{)}
            \PY{c+c1}{\PYZsh{} Filling the matrix from top to bottom, by looping over rows and columns.}
            \PY{k}{for} \PY{n}{i} \PY{o+ow}{in} \PY{n+nb}{range}\PY{p}{(}\PY{n}{m}\PY{o}{+}\PY{l+m+mi}{1}\PY{p}{)}\PY{p}{:}
                \PY{k}{for} \PY{n}{j} \PY{o+ow}{in} \PY{n+nb}{range}\PY{p}{(}\PY{n}{n}\PY{o}{+}\PY{l+m+mi}{1}\PY{p}{)}\PY{p}{:}
                    \PY{k}{if} \PY{n}{i} \PY{o}{==} \PY{l+m+mi}{0}\PY{p}{:}
                        \PY{n}{D}\PY{p}{[}\PY{n}{i}\PY{p}{,} \PY{n}{j}\PY{p}{]} \PY{o}{=} \PY{n}{j}\PY{o}{*}\PY{n}{cost1}  \PY{c+c1}{\PYZsh{} insertion}
                    \PY{k}{elif} \PY{n}{j} \PY{o}{==} \PY{l+m+mi}{0}\PY{p}{:}
                        \PY{n}{D}\PY{p}{[}\PY{n}{i}\PY{p}{,}\PY{n}{j}\PY{p}{]} \PY{o}{=} \PY{n}{i}\PY{o}{*}\PY{n}{cost1}    \PY{c+c1}{\PYZsh{} deletion}
                    \PY{k}{elif} \PY{n}{str1}\PY{p}{[}\PY{n}{i}\PY{o}{\PYZhy{}}\PY{l+m+mi}{1}\PY{p}{]} \PY{o}{==} \PY{n}{str2}\PY{p}{[}\PY{n}{j}\PY{o}{\PYZhy{}}\PY{l+m+mi}{1}\PY{p}{]}\PY{p}{:}
                        \PY{n}{D}\PY{p}{[}\PY{n}{i}\PY{p}{,} \PY{n}{j}\PY{p}{]} \PY{o}{=} \PY{n}{D}\PY{p}{[}\PY{n}{i}\PY{o}{\PYZhy{}}\PY{l+m+mi}{1}\PY{p}{,} \PY{n}{j}\PY{o}{\PYZhy{}}\PY{l+m+mi}{1}\PY{p}{]}
                    \PY{k}{else}\PY{p}{:}
                        \PY{n}{D}\PY{p}{[}\PY{n}{i}\PY{p}{,} \PY{n}{j}\PY{p}{]} \PY{o}{=} \PY{n+nb}{min}\PY{p}{(}\PY{n}{D}\PY{p}{[}\PY{n}{i}\PY{p}{,} \PY{n}{j}\PY{o}{\PYZhy{}}\PY{l+m+mi}{1}\PY{p}{]}   \PY{o}{+} \PY{n}{cost1}\PY{p}{,}    \PY{c+c1}{\PYZsh{} insertion}
                                      \PY{n}{D}\PY{p}{[}\PY{n}{i}\PY{o}{\PYZhy{}}\PY{l+m+mi}{1}\PY{p}{,} \PY{n}{j}\PY{p}{]}   \PY{o}{+} \PY{n}{cost1}\PY{p}{,}    \PY{c+c1}{\PYZsh{} deletion}
                                      \PY{n}{D}\PY{p}{[}\PY{n}{i}\PY{o}{\PYZhy{}}\PY{l+m+mi}{1}\PY{p}{,} \PY{n}{j}\PY{o}{\PYZhy{}}\PY{l+m+mi}{1}\PY{p}{]} \PY{o}{+} \PY{n}{cost2}\PY{p}{)}     \PY{c+c1}{\PYZsh{} substitution}
            \PY{c+c1}{\PYZsh{}Saving the minimal costs in the variable cost.}
            \PY{n}{cost} \PY{o}{=} \PY{n}{D}\PY{p}{[}\PY{n}{m}\PY{p}{,}\PY{n}{n}\PY{p}{]}
            \PY{c+c1}{\PYZsh{} Initialize matrix dimensions to run the backtrace function.}
            \PY{n}{m} \PY{o}{=} \PY{n}{np}\PY{o}{.}\PY{n}{size}\PY{p}{(}\PY{n}{D}\PY{p}{,} \PY{l+m+mi}{0}\PY{p}{)} \PY{o}{\PYZhy{}} \PY{l+m+mi}{1}
            \PY{n}{n} \PY{o}{=} \PY{n}{np}\PY{o}{.}\PY{n}{size}\PY{p}{(}\PY{n}{D}\PY{p}{,} \PY{l+m+mi}{1}\PY{p}{)} \PY{o}{\PYZhy{}} \PY{l+m+mi}{1}
            \PY{c+c1}{\PYZsh{}Getting the backtrace.}
            \PY{n}{changes} \PY{o}{=} \PY{n}{backtrace}\PY{p}{(}\PY{n}{D}\PY{p}{,} \PY{n}{m}\PY{p}{,} \PY{n}{n}\PY{p}{,} \PY{n}{cost1}\PY{p}{,} \PY{n}{cost2}\PY{p}{)}
            \PY{k}{return} \PY{n}{cost}\PY{p}{,} \PY{n}{changes}
\end{Verbatim}


    Computing the edit distance and the backtrace if costs of substitution,
insertion, and deletion equal 1.

    \begin{Verbatim}[commandchars=\\\{\}]
{\color{incolor}In [{\color{incolor}16}]:} \PY{c+c1}{\PYZsh{}Initializing the strings, which shall be analyzed.}
         \PY{n}{X} \PY{o}{=} \PY{l+s+s2}{\PYZdq{}}\PY{l+s+s2}{ACTACTAGATTACTTACGGATCAGGTACTTTAGAGGCTTGCAACCA}\PY{l+s+s2}{\PYZdq{}}
         \PY{n}{Y} \PY{o}{=} \PY{l+s+s2}{\PYZdq{}}\PY{l+s+s2}{TACTAGCTTACTTACCCATCAGGTTTTAGAGATGGCAACCA}\PY{l+s+s2}{\PYZdq{}}
         
         \PY{n+nb}{print}\PY{p}{(}\PY{n}{edit\PYZus{}distance}\PY{p}{(}\PY{n}{X}\PY{p}{,}\PY{n}{Y}\PY{p}{,}\PY{l+m+mi}{1}\PY{p}{,}\PY{l+m+mi}{1}\PY{p}{)}\PY{p}{[}\PY{l+m+mi}{0}\PY{p}{]}\PY{p}{)}
         \PY{n+nb}{print}\PY{p}{(}\PY{n}{edit\PYZus{}distance}\PY{p}{(}\PY{n}{X}\PY{p}{,}\PY{n}{Y}\PY{p}{,}\PY{l+m+mi}{1}\PY{p}{,}\PY{l+m+mi}{1}\PY{p}{)}\PY{p}{[}\PY{l+m+mi}{1}\PY{p}{]}\PY{p}{)}
\end{Verbatim}


\begin{Verbatim}[commandchars=\\\{\}]
{\color{outcolor}Out[{\color{outcolor}16}]:} ['-',
          '-',
          '-',
          '-',
          '-',
          '-',
          '-',
          'S',
          '-',
          'S',
          '-',
          '-',
          '-',
          '-',
          '-',
          '-',
          '-',
          '-',
          '-',
          '-',
          '-',
          '-',
          'S',
          '-',
          '-',
          'S',
          '-',
          '-',
          '-',
          '-',
          '-',
          '-',
          '-',
          '-',
          'S',
          '-',
          '-',
          '-',
          '-',
          '-',
          '-',
          'D',
          'D',
          'D',
          'D',
          'D']
\end{Verbatim}
            
    Computing the edit distance and the backtrace if costs of substitution
equal 1 and the costs of insertion, and deletion equal 2.

    \begin{Verbatim}[commandchars=\\\{\}]
{\color{incolor}In [{\color{incolor}5}]:} \PY{n+nb}{print}\PY{p}{(}\PY{n}{edit\PYZus{}distance}\PY{p}{(}\PY{n}{X}\PY{p}{,}\PY{n}{Y}\PY{p}{,}\PY{l+m+mi}{2}\PY{p}{,}\PY{l+m+mi}{1}\PY{p}{)}\PY{p}{[}\PY{l+m+mi}{0}\PY{p}{]}\PY{p}{)}
        \PY{n+nb}{print}\PY{p}{(}\PY{n}{edit\PYZus{}distance}\PY{p}{(}\PY{n}{X}\PY{p}{,}\PY{n}{Y}\PY{p}{,}\PY{l+m+mi}{2}\PY{p}{,}\PY{l+m+mi}{1}\PY{p}{)}\PY{p}{[}\PY{l+m+mi}{1}\PY{p}{]}\PY{p}{)}
\end{Verbatim}


    \begin{Verbatim}[commandchars=\\\{\}]
15.0
['-', '-', '-', '-', '-', '-', '-', 'S', '-', 'S', '-', '-', '-', '-', '-', '-', '-', '-', '-', '-', '-', '-', 'S', '-', '-', 'S', '-', '-', '-', '-', '-', '-', '-', '-', 'S', '-', '-', '-', '-', '-', '-', 'D', 'D', 'D', 'D', 'D']

    \end{Verbatim}

    Computing the edit distance and the backtrace if costs of substitution,
insertion, and deletion equal 1.

    \begin{Verbatim}[commandchars=\\\{\}]
{\color{incolor}In [{\color{incolor}6}]:} \PY{c+c1}{\PYZsh{}Initializing the strings, which shall be analyzed.}
        \PY{n}{X} \PY{o}{=} \PY{l+s+s2}{\PYZdq{}}\PY{l+s+s2}{AASRPRSGVPAQSDSDPCQNLAATPIPSRPPSSQSCQKCRADARQGRWGP}\PY{l+s+s2}{\PYZdq{}}
        \PY{n}{Y} \PY{o}{=} \PY{l+s+s2}{\PYZdq{}}\PY{l+s+s2}{SGAPGQRGEPGPQGHAGAPGPPGPPGSDG}\PY{l+s+s2}{\PYZdq{}}
        
        \PY{n+nb}{print}\PY{p}{(}\PY{n}{edit\PYZus{}distance}\PY{p}{(}\PY{n}{X}\PY{p}{,}\PY{n}{Y}\PY{p}{,}\PY{l+m+mi}{1}\PY{p}{,}\PY{l+m+mi}{1}\PY{p}{)}\PY{p}{[}\PY{l+m+mi}{0}\PY{p}{]}\PY{p}{)}
        \PY{n+nb}{print}\PY{p}{(}\PY{n}{edit\PYZus{}distance}\PY{p}{(}\PY{n}{X}\PY{p}{,}\PY{n}{Y}\PY{p}{,}\PY{l+m+mi}{1}\PY{p}{,}\PY{l+m+mi}{1}\PY{p}{)}\PY{p}{[}\PY{l+m+mi}{1}\PY{p}{]}\PY{p}{)}
\end{Verbatim}


    \begin{Verbatim}[commandchars=\\\{\}]
37.0
['-', '-', 'S', '-', '-', 'S', 'S', '-', 'S', 'S', 'S', 'S', 'S', 'S', 'S', 'S', '-', '-', 'S', '-', 'S', 'S', 'S', '-', 'S', '-', '-', '-', '-', 'D', 'D', 'D', 'D', 'D', 'D', 'D', 'D', 'D', 'D', 'D', 'D', 'D', 'D', 'D', 'D', 'D', 'D', 'D', 'D', 'D']

    \end{Verbatim}

    Computing the edit distance and the backtrace if costs of substitution
equal 1 and the costs of insertion, and deletion equal 2.

    \begin{Verbatim}[commandchars=\\\{\}]
{\color{incolor}In [{\color{incolor}7}]:} \PY{n+nb}{print}\PY{p}{(}\PY{n}{edit\PYZus{}distance}\PY{p}{(}\PY{n}{X}\PY{p}{,}\PY{n}{Y}\PY{p}{,}\PY{l+m+mi}{2}\PY{p}{,}\PY{l+m+mi}{1}\PY{p}{)}\PY{p}{[}\PY{l+m+mi}{0}\PY{p}{]}\PY{p}{)}
        \PY{n+nb}{print}\PY{p}{(}\PY{n}{edit\PYZus{}distance}\PY{p}{(}\PY{n}{X}\PY{p}{,}\PY{n}{Y}\PY{p}{,}\PY{l+m+mi}{2}\PY{p}{,}\PY{l+m+mi}{1}\PY{p}{)}\PY{p}{[}\PY{l+m+mi}{1}\PY{p}{]}\PY{p}{)}
\end{Verbatim}


    \begin{Verbatim}[commandchars=\\\{\}]
58.0
['-', '-', 'S', '-', '-', 'S', 'S', '-', 'S', 'S', 'S', 'S', 'S', 'S', 'S', 'S', '-', '-', 'S', '-', 'S', 'S', 'S', '-', 'S', '-', '-', '-', '-', 'D', 'D', 'D', 'D', 'D', 'D', 'D', 'D', 'D', 'D', 'D', 'D', 'D', 'D', 'D', 'D', 'D', 'D', 'D', 'D', 'D']

    \end{Verbatim}

    \subsection{Proof of Correctness}\label{proof-of-correctness}

While calculating the given problems along with the added problem
correctly our algorithm has utilized all possible solutions (insert,
delete and substitution). The subproblems and their solutions are finite
and can only be of the following nature: (i-1, j), (i, j-1) and (i-1,
j-1). The algorithm provides a correct solution to the above problems
and it's subproblems showing the full extent of it's functionality as
seen in the above table. We can therefore say that the algorithm can
wield any of the possible solutions successfully. Now when a problem is
larger (longer text strings) we can expect it to deliver the correct
outcome because it is just a matter of having the recursive part of our
algorithm run longer until it reaches the end. We can therefore assume
that it will also prove to be correct for problems of larger size for,
in a longer chain of subproblems (bigger strings of text) the
subproblems will not differ in nature nor will the available solutions
and the algorithm's ability to apply them.

We can equally say that if a person can take an initial step in a
staircase there is nothing technically lacking in order to take the
remaining steps.

    \section{Huffman codes}\label{huffman-codes}

\subsection{Discussion of the probem}\label{discussion-of-the-probem}

    \subsection{The proposed algorithm}\label{the-proposed-algorithm}

    The \textbf{\emph{'frequency\_count'}} function takes a text input
string and creates a dictionary (frequency\_table) of characters found
in the text as keys and their absolute frequency as values. This is
implemented through a for-loop through the text input string that checks
whether the iterative is already a key in the dictionary and if not,
creates it, and if it is, augments its value. The dictionary is then
transformed, by swapping keys and values, which transforms the
dictionary into a list of tuples which are sorted by frequency. This
list of ordered tuples is then returned as output.

The cost of this operation is O(n) because the loop runs through n
operations, and the lookup in the dictionary is constant cost, given
that the length of the alphabet is fixed.

    \begin{Verbatim}[commandchars=\\\{\}]
{\color{incolor}In [{\color{incolor}8}]:} \PY{k}{def} \PY{n+nf}{frequency\PYZus{}count}\PY{p}{(}\PY{n}{text}\PY{p}{)}\PY{p}{:}
            \PY{c+c1}{\PYZsh{} initialise a dictionary \PYZhy{} keys: characters, values: frequency}
            \PY{n}{frequency\PYZus{}table} \PY{o}{=} \PY{p}{\PYZob{}}\PY{p}{\PYZcb{}}
            \PY{c+c1}{\PYZsh{} loop through the string}
            \PY{k}{for} \PY{n}{i} \PY{o+ow}{in} \PY{n}{text}\PY{o}{.}\PY{n}{lower}\PY{p}{(}\PY{p}{)}\PY{p}{:}
                \PY{c+c1}{\PYZsh{} check whether character is already in dictionary}
                \PY{k}{if} \PY{n}{i} \PY{o+ow}{not} \PY{o+ow}{in} \PY{n}{frequency\PYZus{}table}\PY{p}{:}
                    \PY{c+c1}{\PYZsh{} if not in dictionary, add it with frequency 1}
                    \PY{n}{frequency\PYZus{}table}\PY{p}{[}\PY{n+nb}{str}\PY{p}{(}\PY{n}{i}\PY{p}{)}\PY{p}{]} \PY{o}{=} \PY{l+m+mi}{1}
                \PY{k}{else}\PY{p}{:}
                    \PY{c+c1}{\PYZsh{} if it is in dictionary, augment its frequency by 1}
                    \PY{n}{frequency\PYZus{}table}\PY{p}{[}\PY{n+nb}{str}\PY{p}{(}\PY{n}{i}\PY{p}{)}\PY{p}{]} \PY{o}{=} \PY{n}{frequency\PYZus{}table}\PY{p}{[}\PY{n+nb}{str}\PY{p}{(}\PY{n}{i}\PY{p}{)}\PY{p}{]} \PY{o}{+} \PY{l+m+mi}{1}
            \PY{c+c1}{\PYZsh{} swap keys and values, transform into list, and sort the list}
            \PY{n}{frequency\PYZus{}table} \PY{o}{=} \PY{n+nb}{sorted}\PY{p}{(}\PY{p}{[}\PY{p}{(}\PY{n}{v}\PY{p}{,} \PY{n}{k}\PY{p}{)} \PY{k}{for} \PY{n}{k}\PY{p}{,} \PY{n}{v} \PY{o+ow}{in} \PY{n}{frequency\PYZus{}table}\PY{o}{.}\PY{n}{items}\PY{p}{(}\PY{p}{)}\PY{p}{]}\PY{p}{)}
            \PY{k}{return} \PY{n}{frequency\PYZus{}table}
\end{Verbatim}


    The \textbf{\emph{'build\_tree'}} function first calls the
\textbf{\emph{'frequency\_count'}} function on the text input string it
is given, and then builds a Huffman tree from it. The function iterates
through the list from lowest to highest frequency, at each step creating
a new node with the two character tuples with lowest frequency. The node
is created as a nested list with two entries, the sum of the two
children's frequencies, and a nested list with the two tuples. The node
is then inserted back into the list that its entries were extracted
from, with the position chosen by a for loop that compares the nodes
value with each entry from the originator list until the node value is
larger than the iterator's value. This loop runs as long as the top
level list is longer than 1. When it reaches length 1, the tree is
complete and returned as a nested list.

The cost of this operation is constant, because it depends only on the
length of the dictionary given as input, which depends only of the
length of the alphabet and thus is fixed.

    \begin{Verbatim}[commandchars=\\\{\}]
{\color{incolor}In [{\color{incolor}9}]:} \PY{k}{def} \PY{n+nf}{build\PYZus{}tree}\PY{p}{(}\PY{n}{text}\PY{p}{)}\PY{p}{:}
            \PY{c+c1}{\PYZsh{} call frequency\PYZus{}count function and assign to tree}
            \PY{n}{tree} \PY{o}{=} \PY{n}{frequency\PYZus{}count}\PY{p}{(}\PY{n}{text}\PY{p}{)}
            \PY{c+c1}{\PYZsh{} loop over the list, until its length is 1}
            \PY{k}{while} \PY{n+nb}{len}\PY{p}{(}\PY{n}{tree}\PY{p}{)} \PY{o}{\PYZgt{}} \PY{l+m+mi}{1}\PY{p}{:}
                \PY{c+c1}{\PYZsh{} pop out the two lowest value nodes and merge into new node}
                \PY{n}{node} \PY{o}{=} \PY{p}{[}\PY{n}{tree}\PY{p}{[}\PY{l+m+mi}{0}\PY{p}{]}\PY{p}{[}\PY{l+m+mi}{0}\PY{p}{]} \PY{o}{+} \PY{n}{tree}\PY{p}{[}\PY{l+m+mi}{1}\PY{p}{]}\PY{p}{[}\PY{l+m+mi}{0}\PY{p}{]}\PY{p}{,} \PY{p}{[}\PY{n}{tree}\PY{o}{.}\PY{n}{pop}\PY{p}{(}\PY{l+m+mi}{0}\PY{p}{)}\PY{p}{,} \PY{n}{tree}\PY{o}{.}\PY{n}{pop}\PY{p}{(}\PY{l+m+mi}{0}\PY{p}{)}\PY{p}{]}\PY{p}{]}
                \PY{c+c1}{\PYZsh{} check where to place the new node}
                \PY{k}{for} \PY{n}{i} \PY{o+ow}{in} \PY{n+nb}{range}\PY{p}{(}\PY{n+nb}{len}\PY{p}{(}\PY{n}{tree}\PY{p}{)}\PY{p}{)}\PY{p}{:}
                    \PY{c+c1}{\PYZsh{} if the node\PYZsq{}s value \PYZgt{} the current iterate, continue}
                    \PY{k}{if} \PY{n}{node}\PY{p}{[}\PY{l+m+mi}{0}\PY{p}{]} \PY{o}{\PYZgt{}} \PY{n}{tree}\PY{p}{[}\PY{n}{i}\PY{p}{]}\PY{p}{[}\PY{l+m+mi}{0}\PY{p}{]}\PY{p}{:}
                        \PY{k}{pass}
                    \PY{c+c1}{\PYZsh{} if not, assign the iterate\PYZsq{}s value to an index}
                    \PY{k}{else}\PY{p}{:}
                        \PY{n}{index} \PY{o}{=} \PY{n}{i}
                        \PY{k}{break}
                \PY{c+c1}{\PYZsh{} insert the node at the index we found in the for loop}
                \PY{n}{tree}\PY{o}{.}\PY{n}{insert}\PY{p}{(}\PY{n}{index}\PY{p}{,} \PY{n}{node}\PY{p}{)}
            \PY{k}{return} \PY{n}{tree}
\end{Verbatim}


    The \textbf{\emph{'climb\_tree'}} function first calls the
\textbf{\emph{'build\_tree'}} function on the text input it is given,
gets a Huffman tree in the form of a nested list, and then assigns codes
to each node of the tree. The function climbs the tree from its root,
taking left and right steps. While climbing, the function builds a
dictionary with characters from the tree-nodes making up the keys, and
code assignments making up the values. The dictionary is updated at
every step that encounters a leaf of the tree, which is detected with an
if condition that checks whether the node contains a string. If a node
does not contain a string, this means that another list is nested inside
the node, and that we are on a branch rather than a leaf of the tree. In
the case of a branch, the function continues climbing with right and
left steps. Right steps at a 1 to the code string, left steps add a 0 to
the code string.

The cost of this operation is constant, because it depends only on the
length of the nested list it receives as an input. The length and depth
of the nested list in turn depends only on the length of the alphabet,
which, again, is fixed.

    \begin{Verbatim}[commandchars=\\\{\}]
{\color{incolor}In [{\color{incolor}10}]:} \PY{k}{def} \PY{n+nf}{climb\PYZus{}tree}\PY{p}{(}\PY{n}{text}\PY{p}{)}\PY{p}{:}
             \PY{c+c1}{\PYZsh{} call the build\PYZus{}tree function on the text to get a Huffman tree}
             \PY{n}{tree} \PY{o}{=} \PY{n}{build\PYZus{}tree}\PY{p}{(}\PY{n}{text}\PY{p}{)}\PY{p}{[}\PY{l+m+mi}{0}\PY{p}{]}\PY{p}{[}\PY{l+m+mi}{1}\PY{p}{]}
             \PY{c+c1}{\PYZsh{} initialise a code variable to store codes we will assign}
             \PY{n}{code} \PY{o}{=} \PY{n+nb}{str}\PY{p}{(}\PY{p}{)}
             \PY{c+c1}{\PYZsh{} initialise a dictionary that will hold our codes}
             \PY{n}{dictionary} \PY{o}{=} \PY{p}{\PYZob{}}\PY{p}{\PYZcb{}}
         
             \PY{c+c1}{\PYZsh{} define a function that accesses the first element of a node}
             \PY{k}{def} \PY{n+nf}{right\PYZus{}step}\PY{p}{(}\PY{n}{branch}\PY{p}{,} \PY{n}{code}\PY{p}{)}\PY{p}{:}
                 \PY{c+c1}{\PYZsh{} go down to first element}
                 \PY{n}{branch} \PY{o}{=} \PY{n}{branch}\PY{p}{[}\PY{l+m+mi}{0}\PY{p}{]}\PY{p}{[}\PY{l+m+mi}{1}\PY{p}{]}
                 \PY{c+c1}{\PYZsh{} add a 1 to the code string}
                 \PY{n}{code} \PY{o}{=} \PY{n}{code} \PY{o}{+} \PY{n+nb}{str}\PY{p}{(}\PY{l+m+mi}{1}\PY{p}{)}
                 \PY{c+c1}{\PYZsh{} check if we have reached a string rather than a list}
                 \PY{k}{if} \PY{n+nb}{type}\PY{p}{(}\PY{n}{branch}\PY{p}{)} \PY{o}{==} \PY{n+nb}{str}\PY{p}{:}
                     \PY{c+c1}{\PYZsh{} if yes, access the dictionary}
                     \PY{k}{nonlocal} \PY{n}{dictionary}
                     \PY{c+c1}{\PYZsh{} assign the current code to the string found}
                     \PY{n}{dictionary}\PY{p}{[}\PY{n}{branch}\PY{p}{]} \PY{o}{=} \PY{n}{code}
                 \PY{k}{else}\PY{p}{:}
                     \PY{c+c1}{\PYZsh{} if not, take both a right and left step}
                     \PY{n}{right\PYZus{}step}\PY{p}{(}\PY{n}{branch}\PY{p}{,} \PY{n}{code}\PY{p}{)}
                     \PY{n}{left\PYZus{}step}\PY{p}{(}\PY{n}{branch}\PY{p}{,} \PY{n}{code}\PY{p}{)}
                     
             \PY{c+c1}{\PYZsh{} define a fucntion that accesses the second element of a node}
             \PY{k}{def} \PY{n+nf}{left\PYZus{}step}\PY{p}{(}\PY{n}{branch}\PY{p}{,} \PY{n}{code}\PY{p}{)}\PY{p}{:}
                 \PY{c+c1}{\PYZsh{} \PYZsh{} go down to second element}
                 \PY{n}{branch} \PY{o}{=} \PY{n}{branch}\PY{p}{[}\PY{l+m+mi}{1}\PY{p}{]}\PY{p}{[}\PY{l+m+mi}{1}\PY{p}{]}
                 \PY{c+c1}{\PYZsh{} add a 0 to the code string}
                 \PY{n}{code} \PY{o}{=} \PY{n}{code} \PY{o}{+} \PY{n+nb}{str}\PY{p}{(}\PY{l+m+mi}{0}\PY{p}{)}
                 \PY{c+c1}{\PYZsh{} check if we have reached a string rather than a list}
                 \PY{k}{if} \PY{n+nb}{type}\PY{p}{(}\PY{n}{branch}\PY{p}{)} \PY{o}{==} \PY{n+nb}{str}\PY{p}{:}
                     \PY{c+c1}{\PYZsh{} if yes, access the dictionary}
                     \PY{k}{nonlocal} \PY{n}{dictionary}
                     \PY{c+c1}{\PYZsh{} assign the current code to the string found}
                     \PY{n}{dictionary}\PY{p}{[}\PY{n}{branch}\PY{p}{]} \PY{o}{=} \PY{n}{code}
                 \PY{k}{else}\PY{p}{:}
                     \PY{c+c1}{\PYZsh{} if not, take both a right and left step}
                     \PY{n}{right\PYZus{}step}\PY{p}{(}\PY{n}{branch}\PY{p}{,} \PY{n}{code}\PY{p}{)}
                     \PY{n}{left\PYZus{}step}\PY{p}{(}\PY{n}{branch}\PY{p}{,} \PY{n}{code}\PY{p}{)}
                     
             \PY{c+c1}{\PYZsh{} these are the first two steps, which go both left and right}
             \PY{n}{left\PYZus{}step}\PY{p}{(}\PY{n}{tree}\PY{p}{,} \PY{n}{code}\PY{p}{)}
             \PY{n}{right\PYZus{}step}\PY{p}{(}\PY{n}{tree}\PY{p}{,} \PY{n}{code}\PY{p}{)}
             \PY{c+c1}{\PYZsh{} return the dictionary with key: string, value: code.}
             \PY{k}{return} \PY{n}{dictionary}
\end{Verbatim}


    The \textbf{\emph{'encode'}} function calls the
\textbf{\emph{'climb\_tree'}} function to get a dictionary of characters
in the text with their code assignments. It iterates through the input
text string and creates an output text string where every character is
replaced with their code assignment. It then returns the encoded output
string.

The cost of this operation is O(n), because each character needs to be
replaced.

    \begin{Verbatim}[commandchars=\\\{\}]
{\color{incolor}In [{\color{incolor}11}]:} \PY{k}{def} \PY{n+nf}{encode}\PY{p}{(}\PY{n}{text}\PY{p}{)}\PY{p}{:}
             \PY{c+c1}{\PYZsh{} call the climb\PYZus{}tree function to get string\PYZhy{}code assignments}
             \PY{n}{dictionary} \PY{o}{=} \PY{n}{climb\PYZus{}tree}\PY{p}{(}\PY{n}{text}\PY{p}{)}
             \PY{c+c1}{\PYZsh{} initialise output string}
             \PY{n}{encoded\PYZus{}text} \PY{o}{=} \PY{n+nb}{str}\PY{p}{(}\PY{p}{)}
             \PY{c+c1}{\PYZsh{} iterate through text}
             \PY{k}{for} \PY{n}{character} \PY{o+ow}{in} \PY{n}{text}\PY{o}{.}\PY{n}{lower}\PY{p}{(}\PY{p}{)}\PY{p}{:}
                 \PY{c+c1}{\PYZsh{} augment output string with code assignment for each string}
                 \PY{n}{encoded\PYZus{}text} \PY{o}{+}\PY{o}{=} \PY{n}{dictionary}\PY{p}{[}\PY{n}{character}\PY{p}{]}
             \PY{k}{return} \PY{n}{encoded\PYZus{}text}
\end{Verbatim}


    The \textbf{\emph{'decode'}} function requires the same dictionary used
to encode a text string, and the encoded text as input. First, it swaps
the keys and values of the dictionary such that the codes are keys, and
the characters are values. It then iterates through the encoded text and
checks whether it can find an iterate in the dictionary. If not, it adds
the next iterate and checks whether it can find this longer code in the
dictionary. Once it finds a code sequence it can find in the dictionary
it adds the corresponding dictionary entry to the output string. Once
the function has iterated through the entire encoded text it has
recovered the original text string and outputs it.

The cost of this operation is O(n), because each character is recovered
in turn. The length of the encoded string is proportional to the
alphabet used. More complex alphabets will require longer code strings.
Regardless, this increases the cost of the algorithm only by a constant.

    \begin{Verbatim}[commandchars=\\\{\}]
{\color{incolor}In [{\color{incolor}12}]:} \PY{k}{def} \PY{n+nf}{decode}\PY{p}{(}\PY{n}{dictionary}\PY{p}{,} \PY{n}{code}\PY{p}{)}\PY{p}{:}
             \PY{c+c1}{\PYZsh{} get dictionary with string\PYZhy{}code assignments and swap keys and values}
             \PY{n}{dictionary} \PY{o}{=} \PY{n+nb}{dict}\PY{p}{(}\PY{p}{[}\PY{p}{(}\PY{n}{v}\PY{p}{,} \PY{n}{k}\PY{p}{)} \PY{k}{for} \PY{n}{k}\PY{p}{,} \PY{n}{v} \PY{o+ow}{in} \PY{n}{dictionary}\PY{o}{.}\PY{n}{items}\PY{p}{(}\PY{p}{)}\PY{p}{]}\PY{p}{)}
             \PY{c+c1}{\PYZsh{} initialise output string}
             \PY{n}{text} \PY{o}{=} \PY{n+nb}{str}\PY{p}{(}\PY{p}{)}
             \PY{c+c1}{\PYZsh{} initialise pointer for the encoded text}
             \PY{n}{index} \PY{o}{=} \PY{l+m+mi}{0}
             \PY{c+c1}{\PYZsh{} iterate through encoded text,}
             \PY{k}{for} \PY{n}{i} \PY{o+ow}{in} \PY{n+nb}{range}\PY{p}{(}\PY{n+nb}{len}\PY{p}{(}\PY{n}{code}\PY{p}{)}\PY{p}{)}\PY{p}{:}
                 \PY{c+c1}{\PYZsh{} consider expanding encoded text sequence until a match in dict}
                 \PY{k}{if} \PY{n}{code}\PY{p}{[}\PY{n}{index}\PY{p}{:}\PY{n}{i}\PY{o}{+}\PY{l+m+mi}{1}\PY{p}{]} \PY{o+ow}{in} \PY{n}{dictionary}\PY{p}{:}
                     \PY{c+c1}{\PYZsh{} augment output text by the matched string}
                     \PY{n}{text} \PY{o}{=} \PY{n}{text} \PY{o}{+} \PY{n}{dictionary}\PY{p}{[}\PY{n}{code}\PY{p}{[}\PY{n}{index}\PY{p}{:}\PY{n}{i}\PY{o}{+}\PY{l+m+mi}{1}\PY{p}{]}\PY{p}{]}
                     \PY{n}{index} \PY{o}{=} \PY{n}{i} \PY{o}{+} \PY{l+m+mi}{1}
             \PY{k}{return}\PY{p}{(}\PY{n}{text}\PY{p}{)}
\end{Verbatim}


    \begin{Verbatim}[commandchars=\\\{\}]
{\color{incolor}In [{\color{incolor}13}]:} \PY{c+c1}{\PYZsh{} Text examples}
         
         \PY{n}{T1} \PY{o}{=} \PY{l+s+s1}{\PYZsq{}}\PY{l+s+s1}{O all you host of heaven! O earth! What else? And shall I couple }\PY{l+s+s1}{\PYZsq{}}\PYZbs{}
         \PY{l+s+s1}{\PYZsq{}}\PY{l+s+s1}{hell? Oh, fie! Hold, hold, my heart, And you, my sinews, grow not instant }\PY{l+s+s1}{\PYZsq{}}\PYZbs{}
         \PY{l+s+s1}{\PYZsq{}}\PY{l+s+s1}{old, But bear me stiffly up. Remember thee! Ay, thou poor ghost, whiles }\PY{l+s+s1}{\PYZsq{}}\PYZbs{}
         \PY{l+s+s1}{\PYZsq{}}\PY{l+s+s1}{memory holds a seat In this distracted globe. Remember thee! Yea, from }\PY{l+s+s1}{\PYZsq{}}\PYZbs{}
         \PY{l+s+s1}{\PYZsq{}}\PY{l+s+s1}{the table of my memory I’ll wipe away all trivial fond records, All saws }\PY{l+s+s1}{\PYZsq{}}\PYZbs{}
         \PY{l+s+s1}{\PYZsq{}}\PY{l+s+s1}{of books, all forms, all pressures past That youth and observation copied }\PY{l+s+s1}{\PYZsq{}}\PYZbs{}
         \PY{l+s+s1}{\PYZsq{}}\PY{l+s+s1}{there, And thy commandment all alone shall live Within the book and volume }\PY{l+s+s1}{\PYZsq{}}\PYZbs{}
         \PY{l+s+s1}{\PYZsq{}}\PY{l+s+s1}{of my brain, Unmixed with baser matter. Yes, by heaven! O most pernicious }\PY{l+s+s1}{\PYZsq{}}\PYZbs{}
         \PY{l+s+s1}{\PYZsq{}}\PY{l+s+s1}{woman! O villain, villain, smiling, damned villain! My tables! Meet it is }\PY{l+s+s1}{\PYZsq{}}\PYZbs{}
         \PY{l+s+s1}{\PYZsq{}}\PY{l+s+s1}{I set it down That one may smile, and smile, and be a villain. At least }\PY{l+s+s1}{\PYZsq{}}\PYZbs{}
         \PY{l+s+s1}{\PYZsq{}}\PY{l+s+s1}{I’m sure it may be so in Denmark. So, uncle, there you are. Now to my word.}\PY{l+s+s1}{\PYZsq{}}
         
         \PY{n}{T2} \PY{o}{=} \PY{l+s+s1}{\PYZsq{}}\PY{l+s+s1}{Habe nun, ach! Philosophie, Juristerei und Medizin, Und leider auch }\PY{l+s+s1}{\PYZsq{}}\PYZbs{}
         \PY{l+s+s1}{\PYZsq{}}\PY{l+s+s1}{Theologie Durchaus studiert, mit heissem Bemühn. Da steh ich nun, ich armer }\PY{l+s+s1}{\PYZsq{}}\PYZbs{}
         \PY{l+s+s1}{\PYZsq{}}\PY{l+s+s1}{Tor! Und bin so klug als wie zuvor; Heisse Magister, heisse Doktor gar Und }\PY{l+s+s1}{\PYZsq{}}\PYZbs{}
         \PY{l+s+s1}{\PYZsq{}}\PY{l+s+s1}{ziehe schon an die zehen Jahr Herauf, herab und quer und krumm Meine }\PY{l+s+s1}{\PYZsq{}}\PYZbs{}
         \PY{l+s+s1}{\PYZsq{}}\PY{l+s+s1}{Schüler an der Nase herum Und sehe, dass wir nichts wissen können! Das }\PY{l+s+s1}{\PYZsq{}}\PYZbs{}
         \PY{l+s+s1}{\PYZsq{}}\PY{l+s+s1}{will mir schier das Herz verbrennen. Zwar bin ich gescheiter als all die }\PY{l+s+s1}{\PYZsq{}}\PYZbs{}
         \PY{l+s+s1}{\PYZsq{}}\PY{l+s+s1}{Laffen, Doktoren, Magister, Schreiber und Pfaffen; Mich plagen keine }\PY{l+s+s1}{\PYZsq{}}\PYZbs{}
         \PY{l+s+s1}{\PYZsq{}}\PY{l+s+s1}{Skrupel noch Zweifel, Fürchte mich weder vor Hölle noch Teufel Dafür }\PY{l+s+s1}{\PYZsq{}}\PYZbs{}
         \PY{l+s+s1}{\PYZsq{}}\PY{l+s+s1}{ist mir auch alle Freud entrissen, Bilde mir nicht ein, was Rechts zu }\PY{l+s+s1}{\PYZsq{}}\PYZbs{}
         \PY{l+s+s1}{\PYZsq{}}\PY{l+s+s1}{wissen, Bilde mir nicht ein, ich könnte was lehren, Die Menschen zu }\PY{l+s+s1}{\PYZsq{}}\PYZbs{}
         \PY{l+s+s1}{\PYZsq{}}\PY{l+s+s1}{bessern und zu bekehren. Auch hab ich weder Gut noch Geld, Noch Ehr und }\PY{l+s+s1}{\PYZsq{}}\PYZbs{}
         \PY{l+s+s1}{\PYZsq{}}\PY{l+s+s1}{Herrlichkeit der Welt; Es möchte kein Hund so länger leben! Drum hab ich }\PY{l+s+s1}{\PYZsq{}}\PYZbs{}
         \PY{l+s+s1}{\PYZsq{}}\PY{l+s+s1}{mich der Magie ergeben, Ob mir durch Geistes Kraft und Mund Nicht manch }\PY{l+s+s1}{\PYZsq{}}\PYZbs{}
         \PY{l+s+s1}{\PYZsq{}}\PY{l+s+s1}{Geheimnis würde kund; Dass ich nicht mehr mit saurem Schweiss Zu sagen }\PY{l+s+s1}{\PYZsq{}}\PYZbs{}
         \PY{l+s+s1}{\PYZsq{}}\PY{l+s+s1}{brauche, was ich nicht weiss; Dass ich erkenne, was die Welt Im Innersten }\PY{l+s+s1}{\PYZsq{}}\PYZbs{}
         \PY{l+s+s1}{\PYZsq{}}\PY{l+s+s1}{zusammenhält, Schau alle Wirkenskraft und Samen, Und tu nicht mehr in }\PY{l+s+s1}{\PYZsq{}}\PYZbs{}
         \PY{l+s+s1}{\PYZsq{}}\PY{l+s+s1}{Worten kramen.}\PY{l+s+s1}{\PYZsq{}}
\end{Verbatim}


    \begin{Verbatim}[commandchars=\\\{\}]
{\color{incolor}In [{\color{incolor}14}]:} \PY{n}{climb\PYZus{}tree}\PY{p}{(}\PY{n}{T1}\PY{p}{)}
\end{Verbatim}


\begin{Verbatim}[commandchars=\\\{\}]
{\color{outcolor}Out[{\color{outcolor}14}]:} \{'l': '0111',
          'o': '0110',
          '’': '01011111',
          'x': '010111101',
          '?': '010111100',
          'k': '01011101',
          'g': '01011100',
          'w': '010110',
          '.': '0101011',
          'c': '0101010',
          'u': '010100',
          'a': '0100',
          'h': '00111',
          'r': '00110',
          'b': '001011',
          'p': '0010101',
          '!': '0010100',
          'm': '00100',
          'e': '0001',
          'n': '00001',
          's': '00000',
          ' ': '11',
          'f': '101111',
          'v': '101110',
          'y': '10110',
          'i': '1010',
          ',': '10011',
          'd': '10010',
          't': '1000'\}
\end{Verbatim}
            
    \begin{Verbatim}[commandchars=\\\{\}]
{\color{incolor}In [{\color{incolor}15}]:} \PY{n}{climb\PYZus{}tree}\PY{p}{(}\PY{n}{T2}\PY{p}{)}
\end{Verbatim}


\begin{Verbatim}[commandchars=\\\{\}]
{\color{outcolor}Out[{\color{outcolor}15}]:} \{'h': '0111',
          'm': '01101',
          't': '01100',
          'i': '0101',
          'p': '01001111',
          'ü': '01001110',
          'ä': '010011011',
          'q': '0100110101',
          'j': '0100110100',
          ';': '01001100',
          'o': '010010',
          'w': '010001',
          ',': '010000',
          'n': '0011',
          'c': '00101',
          'd': '00100',
          ' ': '000',
          'u': '1111',
          'a': '1110',
          'z': '110111',
          'f': '110110',
          'v': '11010111',
          '!': '11010110',
          '.': '11010101',
          'ö': '11010100',
          'g': '110100',
          's': '1100',
          'e': '101',
          'l': '10011',
          'b': '100101',
          'k': '100100',
          'r': '1000'\}
\end{Verbatim}
            
    \section{References}\label{references}

Huffman, D. (1952). "A Method for the Construction of Minimum-Redundancy
Codes" (PDF). Proceedings of the IRE. 40 (9): 1098--1101.
doi:10.1109/JRPROC.1952.273898.


    % Add a bibliography block to the postdoc
    
    
    
    \end{document}
